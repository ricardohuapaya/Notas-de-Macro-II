\mainsection{5}{Consumo y ahorro}{9/16/2021}

\section{Un Modelo Dinámico de Ahorro y Consumo}

Considere un modelo que se basa en la economía dinámica y micro-fundada: 

\textbf{Requisitos:}

Estamos en un modelo de dos periodos; el \textbf{agente} es un consumidor representativo. Sabemos que no hay dinero ni incertidumbre. Por ahora se sabe que:

\begin{itemize}
    \item El consumidor vive en dos periodos.
    \item El Ingreso es exógeno. $Y_{t}$ por hoy, y $Y_{t+1} $mañana.
    \item El consumidor decide sus niveles de consumó $C_{t}$ y $C_{t+1}$ Puede ahorrar y pedir prestado a una tasa de interés real $r_{t}$
\end{itemize}

\begin{remark}
Asuma por comodidad que todos los consumidores en la economía son idénticos. En este caso los consumidores deben tomar decisiones iguales.
\end{remark}

Las restricciones del consumidores se basan en la decisión de cuanto ahorrar y cuanto consumir. 

\begin{equation}
    C_{t}+S_{t} \leq Y_{t}
\end{equation}

Para el segundo periodo tenemos que: 

\begin{equation}
    C_{t+1}+S_{t+1} \leq Y_{t+1}+ (1+r_{t})S_{t}
\end{equation}

\textbf{Restricciones Presupuestarias de los consumidores:}

Estamos siendo muy generales en la formulación de (5.1) y (5.2); podemos así simplificar las restricciones van de la mano con con saber que siempre se consume todo el ingreso y que no existe ahorro para $t+2$.

\begin{align}
    C_{t}+S_{t} = & Y_{t}\\
    C_{t+1}=& Y_{t+1}+(1+r_{t})S_{t}
\end{align}

El ahorro enlaza en dos periodos donde vemos que: 

$$S_{t}=\frac{C_{t+1}}{1+r_{t}}- \frac{Y_{t+1}}{1+r_{t}}$$

Con ello podemos simplificar el modelo de la forma; 

\begin{equation}
    C_{t}+\frac{C_{t+1}}{1+r_{t}} = Y_{t}+ \frac{Y_{t+1}}{1+r_{t}}
\end{equation}
\myequations{Restriccion presupuestaria intertemporal}

\textbf{Nos dice que el valor presente del flujo de gastos} debe ser igual al flujo de ingresos. 

Asi se expresa graficamente, 
\begin{equation}
    C_{t+1} = Y_{t+1} + (1+r_{t})(Y_{t}-C_{t})
\end{equation}
\textbf{GRAFICO}
\subsection{Curvas de Indiferencia y Derivadas}

Las preferencias de utilidad son conformadas por el consumo de hoy/mañana y la impaciencia.

\begin{equation}
    U(C_{t},C_{t+1}) = u(C_{t})+\beta u(C_{t+1})
\end{equation}

En este caso $U(C_{t},C_{t+1})$ es la función de utilidad total; $u(C_{t})$ es la función de utilidad instantánea. Y luego $\beta$ es un factor de descuento como comparación. 

\textbf{Las preferencias del consumidor del modelo Intertemporal:}

La función de utilidad de el consumidor está conformada 

Una tasa de descuento es diferente a un factor de descuento. Factor de descuento de utilidades $\beta = \frac{1}{1+\rho}$

Note que en este caso las tasas de descuento $\neq$ a los factores de descuento, podemos ver el ejemplo de la forma:
\begin{itemize}
    \item Tasa de interés nominal: $i_{t}$ 
    
    Factor de descuento flujo nominal: $\frac{1}{1-i_{t}}$
    
    \item Tasa de interés real: $r_{t}$ 
    
    Factor de descuento flujo real: $\frac{1}{1-r_{t}}$
    
    \item Tasa de impaciencia: $\rho$ 
    
    Factor de descuento de utilidades:$\beta= \frac{1}{1-\rho}$
\end{itemize}

Ejemplo de funciones de utilidad instantánea

Función Lineal
\begin{equation}
    u(C_{T}) = \theta C_{t}, \theta>0
\end{equation}

Función Cuadrática
\begin{equation}
    u(C_{T}) =C_{t}-\frac{\theta}{2}C^{2}_{t}, \theta>0
\end{equation}

Función Logarítmica
\begin{equation}
    u(C_{T}) =ln C_{t}
\end{equation}

Función Isoelastica
\begin{equation}
    u(C_{T}) =\frac{C_{t}^{1-\sigma}-1}{1-\sigma}, \sigma>0
\end{equation}

Los consumidores enfrentan el problema: 

\begin{align}
    \underset{C_{t}, C_{t+1}}{max} U(C_{t}, C_{t+1})&= u(C_{t})+ \beta u(C_{t+1})\\
    \textup{sujeto a} &\nonumber\\
    C_{t}+\frac{C_{t+1}}{1+r_{t}} &= Y_{t}+\frac{Y_{t+1}}{1+r_{t}}
\end{align}

Despejando la restricción podemos obtener nuevamente (5.6), en este caso nos queda el problema de optimización de la forma; primero tomando que: $U(C_{t}, (Y_{t+1} + (1+r_{t})(Y_{t}-C_{t}))) \equiv V(C_{t})$

\begin{equation}
    \underset{C_{t}}{\max{V(C_{T})}}  = u(C_{t})+\beta u(Y_{t+1} + (1+r_{t})(Y_{t}-C_{t}))
\end{equation}

Si lo solucionamos se requiere que la condición de primer orden sea: 

\begin{align}
    \frac{\partial U}{\partial C_{t}} &= 0 \\
    & \implies u'(C_{t}) - \beta u'((1+r_{t})(Y_{t}-C_{t})+Y_{t+1})((1+r_{t}) = 0\\
    & \implies  u'(C_{t}) = \beta u'((1+r_{t})(Y_{t}-C_{t})+Y_{t+1})((1+r_{t}) 
\end{align}

Podemos resolver el mismo problema con lagaréanos, así tenemos un problema de optimización restringido:
\begin{equation}
    \calL = u(C_{t})+\beta u(C_{t})+\lambda[Y_{t}+Y_{t+1}/(1+r_{t})-C_{t}-C_{t+1}/(1+r_{t})]
\end{equation}
Solucionamos, 

\begin{align}
    \frac{\partial \calL}{\partial C_{t}}  = 0 & \implies \beta u'(C_{t}) =\lambda\\
    \frac{\partial \calL}{\partial C_{t+1}}  = 0 & \implies \beta u'(C_{t+1}) =\lambda/(1+r_{t}) \\
    \frac{\partial \calL}{\partial \lambda}  = 0 & \implies  C_{t}+\frac{C_{t+1}}{1+r_{t}} = Y_{t}+\frac{Y_{t+1}}{1+r_{t}}
\end{align}

Combinando las dos primeras condiciones de primer orden, se tiene la ecuación de Euler

\begin{equation}
    u'(C_{t})=\beta(1+r_{t})u'(C_{t+1})
\end{equation}

Refleja lo que se debe cumplir, la decisión es sopesar qué hacer con la última unidad de ingreso. 

\section{Extensiones}

\subsection{Riqueza}

Iniciamos suponiendo que:

Los consumidores no tienen riqueza y solo hay un activo financiero. Ahora suponemos que los consumidores inician con un acervo de $H_{t-1}$ unidades de riqueza. Casa una se valora en $Q_{t}$

La restricción presupuestaria del consumidor $t$:
\begin{equation}
    C_{t}+S_{t}+Q_{t}H_{t}< Y_{t}+ Q_{t}H_{t-1}
\end{equation}

Alternativamente

\begin{equation}
    C_{t}+S_{t}+Q_{t}\underbrace{(H_{t}-H_{t-1})}_{\Delta H_{t}} \leq Y_{t}
\end{equation}

Considere en (5.24) como la decisión de como decir cuanto cambiar la riqueza. Ahora en el segundo periodo se tiene una restricción similar

\begin{equation}
    C_{t+1}+S_{t+1}+Q_{t+1}\underbrace{(H_{t+1}-H_{t})}_{\Delta H_{t+1}} \leq Y_{t+1}+ (1+r_{t})S_{t}
\end{equation}

Las condiciones generales son, 

$$S_{t+1}=H_{t+1}=0$$

Por lo que la restricicón intertemporal es: 

\begin{equation}
    C_{t}+\frac{C_{t+1}}{1+r_{t}} +Q_{t}H_{t} = Y_{t}+\frac{Y_{t+1}}{1+r_{t}}  + \frac{Q_{t}H_{t-1}+Q_{t+1}H_{t}}{1+r_{t}}
\end{equation}

\begin{itemize}
    \item No hay incertidumbre
    \item \textbf{En equilibrio} el rendimiento de $S_{t}$ y $H_{t}$ debe ser el mismo, Para ilustrar lo previo suponga que, El consumidor solo quiere consumidor mañana:
    $$C_{t}=0$$
    
    El consumidor no tiene riqueza en el periodo pasado,
    
    $$H_{t-1}=0$$
    
    El consumidor no recibe ingreso mañana
    $$Y_{t-1}=0$$
    
    El rendimiento de ahorrar $H_{t}$ es mayor:
    
    $$\frac{Q_{t+1}}{Q_{t}} > 1+r_{t}$$
\end{itemize}

\subsection{Incertidumbre}