\mainsection{1}{Bagaje historico en la Macroeconomia}{8/16/2021}

\textbf{La gente toma decisiones:}
\begin{itemize}
    \item Los recursos disponibles a las personas y sociedades son escasos.
    \item Las personas deciden en la asignación de recursos.
\end{itemize}

Lo anterior da la razón de existir del quehacer económico. Cuando se piensa en la parte macro, pensamos en como cambian las decisiones dependiendo del escenario en el que nos encontremos.

En la \textbf{Metodología de la Economía Positiva}
Friedman (1953), reflexiona sobre el método científico con el que trabaja en economía. No podemos realizar experimentos a gran escala, construir un modelo implica en elegir las fuerzas principales para entender el fenómeno en estudio. 

Recuérdese que un modelo es solo simplificación de la realidad, captura los elementos esenciales para analizar un problema en particular. Estos nos ayudan a entender las relaciones entre variables económicas. 

Así bien \textbf{los fundamentos microeconomicos:}\footnote{Aquí véase la critica de Lucas.}, el comportamiento macroeconómico es la suma de muchas decisiones a nivel micro.

\section{Los Modelos Macroeconomicas}
\textbf{¿Qué es un buen modelo?}

Dicha pregunta no tiene respuesta sencilla, en general se puede desear que: reproduzca las características del fenómeno que se uso para estimarlo. 

Para el análisis de políticas es importante que capture elementos de los datos que no se usaron para estimarlo. 

\textbf{Box (1979)}: "Todos los modelos están equivocados, algunos son útiles." Box desarrolla el \textit{framework} de que la pregunta que \textbf{debe} responder un modelo es:

\begin{center}
    ¿Es este modelo iluminador y útil?
\end{center}

\subsection{Estructura Basica de un Modelo}

\begin{definition}
Podemos plantear que es necesario tener, 
\begin{enumerate}
    \item Consumidores y empresas: como agentes principales
    \item Conjunto de bienes que los consumidores disfrutan
    \item Preferencias de los consumidores sobre los bienes
    \item Tecnología de producción
    \item Disponibilidad de Recursos
\end{enumerate}
\end{definition}

\textbf{Equilibrio:} Los agentes tienen un comportamiento racional, así los consumidores y empresas optimizan.

\textbf{Equilibrio competitivo:} Los consumidores y empresas son tomadores de precios incluyendo a un \textbf{vaciado de mercado.}

\textbf{Variables Exógenas vs Endógenas:}
Así exógenas son determinadas por fuera del modelo, así bien bien las endógenas son determinadas por el modelo. 

Los modelos generan predicciones sobre cómo las variables exógenas afectan las endógenas. 

\section{Historia del Pensamiento Macroeconomico}

El banderazo de salida de la macroeconomía se determina con John Maynard Keynes, quien en \textit{La teoría general del empleo y dinero} da un marco de entendimiento de la macroeconomía. 

La visión keynesiana estableció una perspectiva que podía explicar la gran depresión. Dicho paradigma se fue postulando, no microfundando. 

En el paradigma keynesiano parte de la existencia de rigideces del mercado, estas causan desequilibrio macroeconómicos a corto plazo, dan cabida a intervención del gobierno corregirlos. 

\subsection{Fundamentos de la Teoria Keynesiana}

Cabe resaltar la relación entre consumo, ingreso y efecto multiplicador, es destacable la preferencia por liquidez que explica : 
\begin{itemize}
    \item La demanda de dinero.
    \item El efecto de la política monetaria en tasas de interes y demanda agregada.
\end{itemize}

La importancia de las expectativas de los agentes en la economía, ya que determinan los niveles de consumo e inversión. Así nacen \textit{Espíritus Animales.}

\subsection{La sintesis neoclasica}

Posterior a la \textit{La Teoría General} de Keynes, Samuelson (1955) entra en trabajar por el consenso de incorporar las ideas de Keynes y los clásicos. 

\begin{itemize}
    \item \textbf{Consumo}: Modigliani y Brumberg, presentaron la hipótesis de ciclo de vida, y Friedman presentó la teoría del ingreso permanente.
    \item \textbf{Inversión}: Tobin presento la teoría de $q$ de la inversión.
    \item \textbf{Dinero}: Se trabajó en micro fundamentar la demanda de dinero a partir de decisiones optimizadoras.
    \item \textbf{Crecimiento} Solow (1956) presentó su modelo de crecimiento económico.
\end{itemize}

\subsection{El Monetarismo}

Nace posterior a la Segunda Guerra Mundial, junto con el líder monetarista siendo Friedman estaba escéptico. Durante los 60s se debatieron a los keynesianos en torno a:
\begin{itemize}
    \item La eficacia de la teoría monetaria frente a la política fiscal
    \item Curva de Phillips
    \item El papel de la política macroeconómica. 
\end{itemize}

Los principales argumentos eran que la política monetaria era potente, y así las fluctuaciones en el dinero explican la mayoría de la producción. La mala política monetaria exacerbo inmensamente la Gran Depresión.

Llevó al consenso de que las dos herramientas eran eficaces, tanto el gasto de gobierno como la política monetaria. Los monetaristas dudaban que los economistas supiéramos lo suficiente como para manejar eficaz y oportunamente la economía.

\subsection{La Macroeconomia Microfundada}

Barro, Lucas, Sargent atacaron fuertemente a la teoría keynesiana por ignorar el efecto las expectativas a futuro. Los agentes económicos no tienen previsión perfecta, se pueden equivocar pero no sistemáticamente. 

Si los agentes económicos tienen expectativas racionales:
\begin{itemize}
    \item No se pueden confiar en los modelos macroeconométricos sin microfundamentos
    \item Los tomadores de políticas públicas deben de tomar en cuenta las reacciones de los agentes a sus acciones
\end{itemize}

La herramienta más apta para el análisis de políticas es usando la teoría de juegos para micro-fundamentar el los modelos y así obtener una ganancia de la elección racional de los agentes. 

El periodo de \textbf{los nuevos clásicos y teoría de ciclos reales} de Kydland y Prescott propusieron modelos agregados de equilibrio general. 

\subsection{Macro neokeynesiana}
Dada la critica de Lucas, inician un énfasis de micro-fundar en términos de
\begin{itemize}
    \item Distorsiones reales y nominales
    \item Imperfecciones de mercados laborales
    \item Rigidez de Salario por Contratos
    \item Compt. Perfecta
\end{itemize}

\textbf{Las conclusiones neokeynesianas llegan a ser:}
\begin{itemize}
    \item La política monetaria tiene efectos reales
    \item Se justifica un rol que la política económica estabilice inflación y atenúe fluctuaciones en producción y empleo. 
\end{itemize}

\subsection{Nueva Teoria Crecimiento}

La teoría del crecimiento se estancó después de la década de 1960, asi Lucas posterior desarrolló la nueva teoría
\begin{itemize}
    \item Implicaciones de los retornos crecientes a escala
    \item El rol de la acumulación de Capital Humano
    \item El cambio tech endogeno
    \item Institucionalizad como deter. de crecimiento
\end{itemize}

Así nace la teoría unificada, incluyendo los elementos de fertilidad (Galor), asi llego a explicar la transición histórica de sociedades con muy lento creciente.

