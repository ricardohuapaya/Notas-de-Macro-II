\mainsection{6}{Programación Dinamica}{10/20/2021}

\textbf{El Problema de Generalización de periodos}

Suponga que el hogar vive en el periodo actual, $t$ y $T$ periodos adicionales.

Por lo tanto las $T$ restricciones presupuestarias. Por el momento asuma que que $S_{t+T}=0$ y que las tasas de interés son constantes. Así podemos escribir la restricción presupuestaria intemporal:

\begin{equation}
    C_{t}+\frac{C_{t+1}}{1+r}+\dots+\frac{C_{t+T}}{(1+r)^{T}}=Y_{t}+\frac{Y_{t+1}}{1+r}+\dots+\frac{Y_{t+T}}{(1+r)^{T}}
\end{equation}

Alternativamente, 

\begin{equation}
    \sum_{\tau=0}^{T} \frac{C_{t+\tau}}{(1+r)^{\tau}} =  \sum_{\tau=0}^{T} \frac{Y_{t+\tau}}{(1+r)^{\tau}}
\end{equation}
\myequations{Restricción Presupuestaria T Periodos}

Lo que nos dice es que el flujo de ingreso a valor presente debe ser igual al flujo de consumo en valor presente,

Las preferencias toman una forma similar: 

\begin{equation}
    U(C)=\sum_{\tau =0}^{T}\beta^{\tau}u(C_{t+\tau})
\end{equation}

Esta función demuestra un \textbf{descuento geométrico.}

\textbf{El problema del hogares}

\begin{align}
    \max_{\{C_{\tau}, S_{\tau}\}^{T}_{\tau=t}} & \sum_{\tau =0 }^{T}\beta u(C_{t+\tau}) \\
    \textup{sujeto a }\nonumber \\
    C_{t}+S_{t}&=Y_{t}\\
    C_{t}+S_{t}&=Y_{t}+(1+r_{t}) S_{t})\\
    \dots \nonumber \\
    C_{t+T}+S_{t+T}&=Y_{t+T}+(1+r_{t+T-1}) S_{t+T-1}
\end{align}